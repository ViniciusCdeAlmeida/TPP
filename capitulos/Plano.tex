\newpage
\chapter{Plano pedagógico de curso}

\section{Justificativa da proposta}

Este curso tem como intuito apresentar o básico de conceitos e consultas SQL.

\section{Público-alvo do curso}

Alunos do 1º ano do ensino médio do CED 01 do Cruzeiro das turmas A, B e C.

\section{Inserção social do egresso}

Os estudantes 

\section{Princípios e diretrizes gerais pedagógicas do curso}

\section{Objetivos educacionais e de aprendizagem}

Capacitar o aluno ao entendimento da linguagem de SQL e banco de dados MYSQL.

Aprender técnicas de normatização.
Criar consultas na linguagem SQL para o desenvolvimento a área profissional e para interesse pessoal.
Construir banco de dados.

\section{Delimitação de conhecimentos e saberes (cognitivo, afetivo, psicomotor e  de relação social)}



\section{Mapa de conceitos do domínio cognitivo}



\section{Conteúdo programático}



\newpage
\section{Metodologia e estratégias pedagógicas}

Em todas as aulas serão utilizadas as técnicas de ensino: aula expositiva e interativa, visando sempre a prática do conhecimento.

O curso será desenvolvido entre duas partes distintas:

1° parte da aula: aula expositiva com recursos audiovisuais integrados no computador.
2° parte: aula interativa para aplicabilidade do conteúdo ministrado.

Na disciplina serão feitos exercícios referentes ao conteúdo ministrado no dia, podendo o mesmo ser feito em grupo ou não, a participação do professor será frequente, auxiliando o aprendizado. (A quantidade dependerá do andamento da turma).


\section{Plano de ação de produção e oferta}



\section{Cronograma de produção e oferta}



\section{Infraestrutura de apoio}

\subsection{Infraestrutura física}

Uma sala com 20 computadores, acesso a internet em todos os computadores, quadro branco e projetor.

\subsection{Infraestrutura tecnológica}

Computadores, projetor, internet.

\subsection{Infraestrutura de gestão}

\section{Equipe docente}

Felipe Ken Hirano e Vinícius Corrêa de Almeida

\section{Gestão e apoio pedagógico}

\section{Interação e Comunicação}

\section{Recursos educacionais}

Slides, apostila, ementa e pesquisa na internet.

\section{Estratégias de avaliação}

Os alunos serão avaliados durante e no final do curso com exercícios dura te a aula e um projeto no final. 
