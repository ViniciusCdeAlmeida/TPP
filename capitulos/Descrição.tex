\chapter{Descrição do contexto social e educacional}

Fundado em 20 de janeiro de 1964, através da Resolução 01/64 CD/FEDF, que criou o então Ginásio do Cruzeiro, foi a primeira escola a ser construída no Cruzeiro. Nesta época funcionava apenas no noturno, em um galpão de madeira, com turmas da primeira série ginasial. Em março de 1965, já em prédio próprio e funcionando em regime diurno, passou a denominar-se Ginásio do Cruzeiro – Plano Piloto. Em 03 de janeiro de 1977, através do Decreto 3547/GDF a denominação é alterada para Centro Educacional 01 do Cruzeiro, sendo até hoje conhecido pela comunidade do Cruzeiro como Ginásio. Em 07 de julho de 1980, a Portaria nº 17, da Secretaria de Educação e Cultura passa a ser o documento de reconhecimento da escola.
Em 2008, o turno noturno foi fechado, passando os alunos do Cruzeiro Velho, SMU, S I A e Estrutural a serem atendidos pelo CED 02 do Cruzeiro.
O antigo Ginásio teve seus tempos áureos nas décadas de 70 e 80 por onde passaram pessoas que até hoje fazem parte dos altos escalões do governo do Distrito Federal e da área federal.
Nos últimos anos, até 2011, a escola se encontrava abandonada e cheia de problemas, principalmente pelo desgaste do longo tempo sem uma reforma geral, desestimulando avanços na implementação de projetos e inovações de práticas pedagógicas.
Em 2012, houve uma mudança de espaço físico, quando a escola funcionou no Cruzeiro Novo, em dependências do CED 02 do Cruzeiro, concentrando o Ensino Médio e o Fundamental em um único turno, de forma a permitir que a tão sonhada reforma fosse executada.
Em janeiro de 2013, a comunidade do Cruzeiro velho recebeu as novas instalações da escola. O prédio passou por uma reforma geral com cobertura da quadra, reforma dos laboratórios de Biologia, Física/Química e de informática, espaço mais humanizado para refeitório, sala de recursos e para o SOE, entre outros espaços que contribuirão para planejamentos e construções de propostas de trabalho que melhor atenderão aos anseios da clientela escolar, objetivando um ensino com mais qualidade.
Em fevereiro de 2013, após aprovação dos professores, a escola deu inicio à proposta da semestralidade no Ensino Médio conforme o Currículo em Movimento da SEDF, quarto ciclo, proposta para validação. Neste ano também, após amplo debate, houve a retomada do uso de salas ambientes, os dois turnos.
Em 20 de março de 2013 a escola é reinaugurada em solenidade que contou com a presença de autoridades do Cruzeiro, do GDF e do governador Agnelo.
Atualmente, a escola atende, em sua maioria, alunos da comunidade do Cruzeiro e alunos da Estrutural, mas atende também, embora em menor número, alunos de águas lindas, Gama, Valparaíso, Recanto e outros.

\newpage