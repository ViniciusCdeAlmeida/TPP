\chapter{Descrição do contexto social e educacional}

Fundado em 20 de janeiro de 1964, através da Resolução 01/64 CD/FEDF, que criou oentão Ginásio do Cruzeiro, foi a primeira escola a ser construída no Cruzeiro. Nesta épocafuncionava apenas no noturno, em um galpão de madeira, com turmas da primeira sérieginasial. Em março de 1965, já em prédio próprio e funcionando em regime diurno, passou adenominarse Ginásio do Cruzeiro -Plano Piloto. 

Em 03 de janeiro de 1977, através doDecreto 3547/GDF a denominação é alterada para Centro Educacional 01 do Cruzeiro, sendo até hoje conhecido pela comunidade do Cruzeiro como Ginásio. Em 07 de julho de 1980, aPortaria nº 17, da Secretaria de Educação e Cultura passa a ser o documento dereconhecimento da escola.
Em 2008, o turno noturno foi fechado, passando os alunos do Cruzeiro Velho, SMU, SI A e Estrutural a serem atendidos pelo CED 02 do Cruzeiro.

O antigo Ginásio teve seus tempos áureos nas décadas de 70 e 80 por onde passaram pessoas que até hoje fazem parte dos altos escalões do governo do Distrito Federal e da área federal.

Nos últimos anos, até 2011, a escola se encontrava abandonada e cheiade problemas, principalmente pelo desgaste do longo tempo sem uma reforma geral, desestimulando avanços na implementação de projetos e inovações de práticas pedagógicas.

Em 2012, houve uma mudança de espaço físico, quando a escola funcionou no Cruzeiro Novo, em dependências do CED02 do Cruzeiro, concentrando o Ensino Médio e o Fundamental em um único turno, de forma a permitir que a tão sonhada reforma fosse executada.

Em janeiro de 2013, a comunidade do Cruzeiro velho recebeu as novas instalações da escola. O prédio passou por uma reforma geral com cobertura da quadra, reforma dos laboratórios de Biologia, Física/Química e de Informática, espaços mais humanizados para o refeitório, sala de recursos e para o SOE, entre outros espaços que contribuirão aos anseios da clientela escolar, objetivando um ensino com mais qualidade.


Em fevereiro de 2013, após aprovação dos professores, a escola deu inicio à propostada semestralidade no Ensino Médio conforme o Currículo em Movimento da SEDF, quartociclo, proposta para validação. Neste ano também, após amplo debate, houve a retomada douso de salas ambientes, os dois turnos.

Em 20 de março de 2013 a escola é reinaugurada em solenidade que contou com apresença de autoridades do Cruzeiro, do GDF e do governador Agnelo.
O CED 01 do Cruzeiro adotou, em 2013, o Regime Anual com Blocos Semestrais, a partir de discussão com a comunidade em busca de estratégias que melhorasse o rendimento escolar e a aprendizagem, visando diminuir a repetência e estimulando a permanência dos estudantes na UE, promovendo igualdade social e educacional.

Em 2014, com a possibilidade da oferta da Educação Profissional integrada ao Ensino Médio, iniciou-se o debate sobre os seguintes temas: a relação da formação profissional com a comunidade na perspectiva de intervenção social, a estrutura física e o material humano disponível na unidade de ensino e a vocação dos estudantes. A primeira decisão do grupo foi a de oferecer curso técnico e manter a modalidade de regime anual com blocos semestrais.

No CED 01 do Cruzeiro, mais de 50% dos estudantes são da Cidade Estrutural. Boa parte da população desta comunidade  não tem garantido seus direitos básicos de acesso a saneamento básico, iluminação pública, educação na localidade onde residem e acesso à formação profissional.

A discussão sobre a oferta integrada prosseguiu em coordenação pedagógica. Os professores do turno vespertino, que trabalham com as turmas dos anos finais do Ensino Fundamental estavam conscientes de que a escola não poderia perder a oportunidade de oferecer o Ensino Médio Integrado à Educação Profissional – EMI a essa comunidade. Todos sabem que a decisão pelo EMI ocasionará o remanejamento das turmas de Ensino Fundamental para outra escola. Na reunião seguinte com todos os turnos da escola foi aprovada a adesão da mesma ao Ensino Médio Integrado à Educação Profissional, com perfil técnico em Informática, com o seguinte resultado: 19 (dezenove) votos favoráveis, 03 (três) abstenções e 04 (quatros) votos para o perfil técnico em Jogos Digitais.

Iniciou-se então nova pesquisa junto a comunidade escolar sobre qual curso adotar, considerando o perfil informática e o Catálogo Nacional de Cursos Técnicos.

Em reunião com a presença de todos os profissionais vinculados à escola, foi apresentado o resultado da pesquisa feita pela coordenação pedagógica, envolvendo toda a instituição, que terminou com a opção pelo eixo Informação e Comunicação. Um novo encontro que contou com a presença de um profissional de Informática, com experiência na formação e no mercado de trabalho foi fundamental para a definição do perfil do curso: Técnico em Informática para Internet.

O Técnico em Informática para Internet é responsável por desenvolver programas de computador para internet, seguindo as especificações e paradigmas da lógica e das linguagens de programação. Utiliza ferramentas de desenvolvimento de sistemas, para construir soluções que auxiliam o processo de criação de interfaces e aplicativos empregados no comércio e marketing eletrônico. Além disso, desenvolve e realiza a manutenção de sites e portais na internet.

O mercado de trabalho para este profissional está nas instituições públicas, privadas e do terceiro setor que demandam pessoas preparadas para as novas tecnologias da informática para programação de computadores para internet com olhar voltado aos sistemas de códigos abertos (LINUX) e às novas possibilidades da internet.

Durante todo esse percurso, a escola foi apoiada pela Coordenação de Educação Profissional, Coordenação de Ensino Médio e Coordenação Regional do Plano Piloto Cruzeiro (Gerência de Educação Básica), que elaborou calendário de encontros para suporte técnico que ocorreram nos horários de coordenação pedagógica e o envolvimento da Subsecretaria de Educação Básica.

EQUIPE GESTORA:

Diretor: Jovandir Botelho de Andrade

Vice-diretor: Getúlio Sousa Cruz

Chefe de secretaria: Beneval Diuza da Silva Júnior 

Supervisor Administrativo: Helder Sousa Martins

Coordenadores: 	Humbertânio Hilário da Silva;
Ana Carolina Cabral;
Eudes Henrique da Silva;
Maria de Deus Coelho R. Faria;
Marina Duarte Teixeira;
Júlio Pedro Soares de Oliveira;
Jairo Joaquim Neres.

Orientadora: Márcia Maria de Mendonça Coimbra.

\newpage